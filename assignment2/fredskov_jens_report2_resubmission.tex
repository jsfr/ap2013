\documentclass[a4paper, 11pt]{article}
\usepackage[utf8x]{inputenc}
\usepackage[T1]{fontenc}
\usepackage{ucs}
\usepackage[english]{babel}
\usepackage{lmodern}
\usepackage{mathtools, amsfonts}
\usepackage[parfill]{parskip}
\usepackage{graphicx, float}

\usepackage{fancyhdr} % Custom headers and footers
\pagestyle{fancyplain} % Makes all pages in the document conform to the custom headers and footers
\fancyhead{} % No page header - if you want one, create it in the same way as the footers below
\fancyfoot[L]{} % Empty left footer
\fancyfoot[C]{} % Empty center footer
\fancyfoot[R]{\thepage} % Page numbering for right footer
\renewcommand{\headrulewidth}{0pt} % Remove header underlines
\renewcommand{\footrulewidth}{0pt} % Remove footer underlines
\setlength{\headheight}{13.6pt} % Customize the height of the header

\widowpenalty=1000
\clubpenalty=1000

\newcommand{\horrule}[1]{\rule{\linewidth}{#1}} % Create horizontal rule command with 1 argument of height

\title{ 
\normalfont\normalsize 
\textsc{University of Copenhagen} \\ [25pt] % Your university, school and/or department name(s)
\horrule{0.5pt} \\[0.4cm] % Thin top horizontal rule
\huge Assignment 2: Counting parentheses \\ \Large - resubmission % The assignment title
\horrule{2pt} \\[0.5cm] % Thick bottom horizontal rule
}

\author{Jens Fredskov (chw752)} % Your name

\date{\normalsize\today} % Today's date or a custom date

\begin{document}
\maketitle

\section{Introduction} % (fold)
\label{sec:introduction}

The following report describes the implementation and testing of \emph{let} and \emph{lambda} in Haskell library for interpreting the APLisp dialect. 

% section introduction (end)

\section{Implementation} % (fold)
\label{sec:implementation}

\subsection{Let} % (fold)
\label{sub:let}

\emph{let} is implemented in \texttt{eval}. If the scope is empty we evaluate tyhe body. If there actually are variables in the scope we evaluate each of their values and using \texttt{bindVar} with \texttt{local} binds it in the a local environment. Finally we again evaluate the body using the new environment.

% subsection let (end)

\subsection{Lambda} % (fold)
\label{sub:lambda}

In \texttt{eval} \emph{lambda} is merely returned. This means that we can store a \emph{lambda} in a \emph{let} and then apply it using \texttt{funcall}. This is done in \texttt{apply} where we allow either a single argument or a list of arguments. \texttt{apply} sends its arguments to \texttt{evalLambda} which binds every parameter to the respective argument of in a local scope and then evaluates the body. If there is a mismatch between arguments and parameters an error is returned.

% subsection lambda (end)

% section implementation (end)

\section{Testing} % (fold)
\label{sec:testing}

Extra test cases have been added in \texttt{APLispTest.hs} to test \emph{let} and \emph{lambda}, all of which returns the expected results.

% section testing (end)

\section{Conclusion} % (fold)
\label{sec:conclusion}

The implementation of \emph{let} and \emph{lambda} have been described. The tests have been documented and all return the expected result.

% section conclusion (end)

\end{document}